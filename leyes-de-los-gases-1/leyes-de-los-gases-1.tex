\documentclass{article}
\usepackage{mathtools}
\usepackage{amsthm}
\usepackage{geometry}

\title{Leyes de los Gases 1}
\author{Edgar Trejo Avila}
\date{Sábado 22 de octubre del 2022}

% https://www.overleaf.com/learn/latex/Page_size_and_margins
\geometry{bmargin=1in,tmargin=1in}

\begin{document}
\maketitle

La ley de \textit{Boyle} fue uno de los modelos
matemáticos más relevantes propuestos para entender
el comportamiento de los gases.

La ley de Boyle establece que el producto de la presión
y el volumen de un gas en un proceso \textbf{isotérmico}
es una constante, es decir, que dichas variables son 
inversamente proporcionales y son iguales para cualquier
configuración del gas:

\begin{align*}
    P_0V_0 = PV = C \\
    P = \frac{C}{V} \\
    V = \frac{C}{P}
\end{align*}

\end{document}
