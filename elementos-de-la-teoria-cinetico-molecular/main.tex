\documentclass{article}
\usepackage{mathtools}
\usepackage{amsthm}
\usepackage{geometry}

\title{Elementos de la Teoría Cinético-Molecular}
\author{Edgar Trejo Avila}
\date{Jueves 10 de octubre del 2022}

% https://www.overleaf.com/learn/latex/Page_size_and_margins
\geometry{bmargin=1in,tmargin=1in}

\begin{document}
\maketitle

En el siglo XIX, Ludwing Boltzmann demostró que la energía
cinética media \(k_m\) de las moléculas (i. e., un aproximado
de la energía cinética de cada una) es directamente 
proporcional a la temperatura absoluta \(T\):

\begin{align}
    k_m = \frac{3kT}{2} \label{energyone} \\
    T = \frac{2k_m}{3k} \nonumber
\end{align}

Donde \(k\) es la constante de Boltzmann:

\[
    k = 1.38\times10^{-23} \frac{J}{K}
\]

Igualando (\ref{energyone}) con la fórmula usual
de la energía cinética (usando variables dentro 
del contexto de una sola molécula), se obtiene:

\begin{align}
    \frac{mv^2}{2} = \frac{3kT}{2} \nonumber \\
    v = \sqrt{\frac{3kT}{m}}
\end{align}

\end{document}
