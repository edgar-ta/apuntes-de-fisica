\documentclass{article}
\usepackage{mathtools}
\usepackage{amsthm}
\usepackage{geometry}

\title{Elementos de la Teoría Cinético Molecular 1}
\author{Edgar Trejo Avila}
\date{Sábado 22 de octubre del 2022}

% https://www.overleaf.com/learn/latex/Page_size_and_margins
\geometry{bmargin=1in,tmargin=1in}

\begin{document}
\maketitle

\section*{Una Breve Historia del Mol}

Partimos de la idea de que, los químicos de la época
necesitaban una manera formal de medir porciones de sustancias
(i. e., una pauta respecto a la cual tomar sus mediciones);
particularmente, se necesitaba una unidad que definiera
la cantidad de moléculas a utilizar por medición, pues era 
de gran importancia el operar a nivel molecular para poder 
definir con precisión ciertas propiedades químicas de los elementos.

Llamemos a esa unidad \textit{mol}; dado que su equivalencia
en moléculas es aún indefinida, no se pueden hacer cálculos con
ella, sin embargo, es posible experimentar con sus características
para lograrlo. Digamos pues que el \textit{mol} se vuelve una
unidad común en la química y se tienen mediciones de propiedades
de sustancias usándola; en específico, la masa \(M\) de un \textit{mol}
de moléculas de sustancia, y la masa \(n\) de cada una de las moléculas:

\begin{align*}
    M &::= \frac{kg}{mol}\\
    n &::= \frac{kg}{molecula}
\end{align*}

Resulta entonces que podemos calcular el número \(N\) de moléculas por
\(mol\) haciendo el siguiente cálculo:

\begin{align*}
    N \frac{molecula}{mol} &= \frac{M \frac{kg}{mol}}{n \frac{kg}{molecula}}\\
    &= \frac{M}{n} \frac{molecula}{mol}
\end{align*}

Nótese como \(N\) es una constante; la forma en que está definida
es tal que la equivalencia de un \(mol\) en moléculas es la misma
independientemente de la sustancia, i. e., un \(mol\) de hierro 
debe tener la misma cantidad de moléculas que un \(mol\) de carbono,
por ejemplo. Dado que \(N\) es una constante, solo hace falta medirla
una vez para dejarla definida para siempre; lo que hicieron los 
químicos de la época fue tomar la masa molecular (i. e., la masa 
de una sola molécula) de una sustancia conocida para la época y después inventar
la masa que un \(mol\) de esa sustancia debería tener (si apenas se 
está definiendo el \(mol\) no se puede tomar una medida real que 
dependa de él; tuvo que haber entonces un factor de arbitrariedad en
el proceso).

La sustancia elegida fue el carbono, que se sabe tiene una masa de \(1.66\times10^{-27}\frac{kg}{molecula}\),
y la masa elegida fue de \(1\frac{kg}{mol}\):

\begin{align*}
    N\frac{molecula}{mol} &= \frac{1\frac{kg}{mol}}{1.66 \times 10^{-27}\frac{kg}{molecula}}\\
    &= 6.02 \times 10^{26} \frac{molecula}{mol}
\end{align*}

Puesto verbalmente:

\begin{quote}
    Un mol está definido como la cantidad de moléculas
    de carbono que hay en un kilogramo de esa sustancia
\end{quote}

El descubrimiento de dicho número \(N\) se le atribuye al físico-químico
italiano \textit{Amadeo Avogadro}, y por eso es que se le conoce
como \textit{el número de Avogadro}:

\begin{align}
    N_A = 6.02 \times 10^{26} \frac{molecula}{mol} \label{orignalAvogadro}
\end{align}

Ahora bien, uno de los propósitos del \(mol\) es facilitar
la medición de propiedades al nivel molecular; esto es
posible pues el número de moléculas que representa es tan grande
que la cantidad de sustancia a medir se vuelve tangible y
realizable aún sin tanta precisión (i. e., un grupo de 
\(6.02 \times 10^{26}\) moléculas de sustancia forman una masa
que se puede ponderar con una báscula casera, a diferencia de, dígase,
\(100\) moléculas, que requerirían de equipo muy especializado 
para medirlas satisfactoriamente).

Sin embargo, aún siendo un número tan grande, los químicos 
pronto se dieron cuenta de que el \(mol\) se quedaba 
corto en varias ocaciones, así que decidieron deprecar su uso y migrar
a una unidad derivada; el \(kmol\) (equivalente, por supuesto,
a \(1,000\,moles\)). Ajustando la ecuación (\ref{orignalAvogadro}) a \(kmoles\),
se tiene que:

\[
    N_A = 6.02 \times 10^{29} \frac{molecula}{kmol}
\]


\paragraph*{Ecuación Fundamental de la Teoría Cinético-Molecular}

\end{document}
