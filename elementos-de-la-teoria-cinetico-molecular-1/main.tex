\documentclass{article}
\usepackage{mathtools}
\usepackage{amsthm}
\usepackage{geometry}

\title{Elementos de la Teoría Cinético Molecular 1}
\author{Edgar Trejo Avila}
\date{Sábado 22 de octubre del 2022}

% https://www.overleaf.com/learn/latex/Page_size_and_margins
\geometry{bmargin=1in,tmargin=1in}

\begin{document}
\maketitle

\section*{Una Breve Historia del Mol}

Hubo un momento en que la química seguía siendo una
ciencia informal (de hecho, lleva poco tiempo de ser formalizada
al compararla con ciencias como la física o las matemáticas)

Partimos de la idea de que, los químicos de la época
necesitaban una manera de medir propiedades de sustancias
(i. e., una pauta respecto a la cual tomar sus mediciones);
particularmente, se necesitaba una unidad que definiera
la cantidad de moléculas a utilizar por medición, pues era
de gran importancia el operar a nivel molecular para poder
definir con precisión ciertas propiedades químicas de los elementos.

Llamemos a esa unidad \textit{mol}; dado que su equivalencia
en moléculas es aún indefinida, no se pueden hacer cálculos con
ella, sin embargo, es posible experimentar con sus características
para lograrlo. Digamos pues que el \textit{mol} se vuelve una
unidad común en la química y se tienen mediciones usándola;
en específico, la masa \(M\) de un \textit{mol}
de moléculas de sustancia, y la masa \(n\) de cada una de las moléculas:

\begin{align*}
    M & ::= \frac{kg}{mol}      \\
    n & ::= \frac{kg}{molecula}
\end{align*}

Resulta entonces que podemos calcular el número \(N\) de moléculas por
\(mol\) haciendo el siguiente cálculo:

\begin{align}
    N \frac{molecula}{mol} & = \frac{M \frac{kg}{mol}}{n \frac{kg}{molecula}} \nonumber
    \\
                           & = \frac{M}{n} \frac{molecula}{mol} \label{avogadroEquivalence}
\end{align}

Nótese como \(N\) es una constante; la forma en que está definida
es tal que la equivalencia de un \(mol\) en moléculas es la misma
independientemente de la sustancia, i. e., un \(mol\) de hierro
debe tener la misma cantidad de moléculas que un \(mol\) de carbono,
por ejemplo. Dado que \(N\) es una constante, solo hace falta medirla
una vez para dejarla definida para siempre; lo que hicieron los
químicos de la época fue tomar la masa molecular (i. e., la masa
de una sola molécula) de una sustancia conocida y después inventar
la masa que un \(mol\) de esa sustancia debería tener (si apenas se
está definiendo el \(mol\) no se puede tomar una medida real que
dependa de él; tuvo que haber entonces un factor de arbitrariedad en
el proceso).

La sustancia elegida fue el carbono, que se sabe tiene una masa de \(1.66\times10^{-27}\frac{kg}{molecula}\),
y la masa elegida fue de \(1\frac{kg}{mol}\):

\begin{align*}
    N\frac{molecula}{mol} & = \frac{1\frac{kg}{mol}}{1.66 \times 10^{-27}\frac{kg}{molecula}} \\
                          & = 6.02 \times 10^{26} \frac{molecula}{mol}
\end{align*}

Puesto verbalmente:

\begin{quote}
    Un mol se define como la cantidad de moléculas
    de carbono que hay en un kilogramo de esa sustancia
\end{quote}

El descubrimiento de dicho número \(N\) se le atribuye al físico-químico
italiano \textit{Amadeo Avogadro}, y por eso es que se le conoce
como el \textit{número de Avogadro} (representado como \(N_A\)):

\begin{align}
    N_A = 6.02 \times 10^{26} \frac{molecula}{mol} \label{originalAvogadro}
\end{align}

Ahora bien, uno de los propósitos del \(mol\) es facilitar
la medición de propiedades al nivel molecular; esto es
posible pues el número de moléculas que representa es tan grande
que la cantidad de sustancia a medir se vuelve tangible y
realizable aún sin tanta precisión (i. e., un grupo de
\(6.02 \times 10^{26}\) moléculas de sustancia forman una masa
que se puede ponderar con una báscula casera, a diferencia de, dígase,
\(100\) moléculas, que requerirían de equipo muy especializado
para medirlas satisfactoriamente).

Sin embargo, aún siendo un número tan grande, los químicos
pronto se dieron cuenta de que el \(mol\) se quedaba
corto en varias ocaciones, así que decidieron reemplazar su uso migrando
a una unidad derivada; el \(kmol\) (equivalente, por supuesto,
a \(1,000\,moles\)). Ajustando la ecuación (\ref{originalAvogadro}) a \(kmoles\),
se tiene que:

\[
    N_A = 6.02 \times 10^{29} \frac{molecula}{kmol}
\]

Retomando lo expuesto en (\ref{avogadroEquivalence}) y ajustando también a \(kmoles\):

\begin{align}
    n \frac{kg}{molecula} = \frac{M \frac{kg}{kmol}}{ N_A \frac{molecula}{kmol}}
\end{align}

La ecuación anterior es bastante útil, pues permite determinar una propiedad de una
sustancia a nivel molecular sin requerir de instrumentos de precisión irreal, siguiendo
con el que se ha dicho es el propósito de la propia creación del \(mol\).

\section*{El Número de Loschmidt}

Es un hecho conocido que, en un proceso \textbf{isotérmico} e \textbf{isobárico},
el volumen de un \(kmol\) de gas es constante; de \(22.4\,m^3\):

\[
    C = 22.4 \frac{m^3}{kmol}
\]

Utilizando el \textit{número de Avogadro}, uno
puede calcular el número de moléculas \(n\) por
unidad de volumen:

\begin{align*}
    n \frac{molecula}{m^3} & = \frac{6.02 \times 10^{26} \frac{molecula}{kmol}}{22.4 \frac{m^3}{kmol}} \\
                           & = 26.81 \times 10^{24} \frac{molecula}{m^3}
\end{align*}

Este número se le atribuye al científico austriaco
\textit{Josef Loschmidt}, y se le conoce como el
\textit{número de Loschmidt} (representado como \(n_L\)):

\[
    n_L = 26.81 \times 10^{24} \frac{molecula}{m^3}
\]

\section*{El Cubo de Gas}

Medir propiedades de los gases siempre ha sido un reto
para los científicos; no se pueden pesar como uno haría
con un sólido o un líquido, no se puede medir su volumen,
y variaciones pequeñas de temperatura/presión afectan
significativamente su comportamiento. Sin embargo, aún hay
esperanza de dominar a este estado de la materia tan peculiar;
queda claro que los gases son bastante dependientes a las
condiciones del ambiente en el que se les mide, entonces,
la mejor forma de obtener una medición acertada es controlando
el ambiente tanto como sea posible (de ahí que las fórmulas
relacionadas a los gases constantemente aclaren condiciones
como "a temperatura ambiente", "a presión atmosférica",
"con un volumen constante" y demás).

La forma más común para realizar mediciones en los gases es
aislarlos en una cámara de vacío (algo parecido a lo que se
hace para pesarlos); consideremos pues una cámara en forma de cubo
albergando un gas y experimentemos con sus características para
calcular una de sus propiedades físicas, en este caso, la presión
ejercida por el gas en las caras de la cámara.

Como condiciones iniciales, se sabe que el cubo tiene una longitud
de \(L\) metros y alberga \(n\) moléculas moviéndose a distintas
velocidades, más aún, asúmase que las moléculas están rebotando
constantemente de una cara del cubo a otra (la contraria en el mismo eje);
si se sabe la cantidad \(n'\) de moléculas que están rebotando en un eje,
uno puede calcular la presión ejercida en las caras de dicho eje
al conocer la fuerza producida por las moléculas (haría falta conocer
también el área de cada cara, pero eso es trivial pues ya se tiene
el lado del cubo/cámara). Para calcular la fuerza total de las moléculas,
considérese primero el comportamiento de una sola, déjese fija una \(molecula_i\)
con velocidad \(v_i\) y masa \(m_i\), se tiene que:

\[
    p_i = v_i m_i
\]

Donde \(p_i\) es el ímpetu inicial de la \(molecula_i\); dado que ésta
está rebotando constantemente, eventualmente golpeará la otra cara el
mismo eje y regresará al lugar de partida con una velocidad negativa
(si la \(molecula_i\) parte siguiendo una dirección positiva,
al rebotar la cambiará por la dirección contraria que, por definición,
es negativa); entonces se tiene que:

\[
    p'_i = -v_i m_i
\]

Donde \(p'_i\) es el ímpetu final de la \(molecula_i\); considérese ahora
el tiempo \(t_i\) que le toma llegar a la posición final desde la
posición incial. Se sabe que \(t = \frac{d}{v}\), y que la \(molecula_i\)
tendrá que recorrer una distancia de \(2L\) (la longitud de un lado,
una vez para la ida y otra para la vuelta) entonces:

\[
    t_i = \frac{2L}{v_i}
\]

Luego, es un hecho conocido que \(F = \frac{\Delta p}{\Delta t}\), entonces,
la fuerza \(F_i\) de la \(molecula_i\) es tal que:

\begin{align*}
    F_i & = \frac{\Delta p}{\Delta t}                 \\
        & = \frac{p'_i - p_i}{\frac{2L}{v_i} - 0}     \\
        & = \frac{-v_i m_i - v_i m_i}{\frac{2L}{v_i}} \\
        & = \frac{-2v_i m_i}{\frac{2L}{v_i}}          \\
        & = -\frac{v^2_i m_i}{L}
\end{align*}

Nótese que el signo negativo viene dado por el marco
de referencia respecto al cual se mide la velocidad;
como tal, no tiene sentido físico decir que la
fuerza es negativa si no se considera ese mismo
marco y, por lo tanto, se puede depreciar:

\[
    F_i = \frac{v^2_i m_i}{L}
\]

Ahora, esta fórmula se puede aplicar para cada una
de las \(n'\) moléculas rebotando en un eje, entonces
la fuerza total \(F_{total}\) es tal que:

\begin{align*}
    F_{total} & = \sum_{i = 1}^{n'}  F_i                \\
              & = \sum_{i = 1}^{n'} \frac{v^2_i m_i}{L} \\
\end{align*}

En la teoría molecular, se considera que propiedades 
físicas como la masa son iguales para cada molécula 
de una misma sustancia, entonces:

\begin{align*}
    F_{total} = \frac{m}{L} \sum_{i = 1}^{n'} v^2_i
\end{align*}

La sumatoria de la ecuación anterior requiere 
conocimientos/herramientas de medición que se salen
del ámbito de la física, por lo que es mejor 
relegársela a una ciencia más apta, encargada de 
la toma y el análisis de datos; la estadística. Ésta
nos provee de un cálculo bastante similar al de 
la sumatoria anterior, conocido como desviación 
estándar, en la forma de una variable \(v_{cm}\)
(leída como "velocidad cuadrática media"):

\[
    v_{cm} = \sqrt{\frac{\sum_{i = 1}^{n'} v_i^2}{n'}}
\]

Esto es:

\begin{align*}
    v_{cm}^2    &= \frac{\sum_{i = 1}^{n'} v_i^2 }{n'},\\
    v_{cm}^2 n' &= \sum_{i = 1}^{n'} v_i^2
\end{align*}

Por lo tanto:

\begin{align*}
    F_{total}   &= \frac{m}{L} v_{cm}^2 n' \\
                &= \frac{m n' v_{cm}^2}{L}
\end{align*}

Ahora, considérese que las partículas dentro de la 
cámara/cubo están repartidas equitativamente en sus 
ejes (recuérdese que un cubo tiene tres ejes), entonces:

\[
    n' = \frac{n}{3}
\]

Esto es:

\[
    F_{total} = \frac{m n v_{cm}^2}{3 L}
\]

Se sabe que la presión está dada por la fórmula \(P = \frac{F}{A}\)
y, como se detalló al inicio, calcular el área de la cara
del cubo es trivial; entonces:

\begin{align*}
    P   &= \frac{F_{total}}{A} \\
        &= \frac{\frac{m n v_{cm}^2}{3 L}}{L^2} \\
        &= \frac{m n v_{cm}^2}{3 L^3}
\end{align*}

Dado que la expresión \(L^3\) es igual al volumen \(V\) del cubo,
se tiene que:

\begin{align*}
    P = \frac{m n v_{cm}^2}{3 V}
\end{align*}

Nótese que el producto \(m n\) es equivalente a la masa
dentro del cubo (en efecto pues, por definición, éste es el producto
de la masa de cada molécula por el número de moléculas dentro
del cubo); más aún, al dividirlo por el volumen se obtiene la 
densidad \(\rho\) del mismo, de modo que:

\begin{align}
    P = \frac{\rho v_{cm}^2}{3}
\end{align}

Si se deja \(n_0\) igual al número de moléculas por unidad 
de volumen, \(\rho\) se puede reescribir como \(m n_0\), entonces:

\[
    P = \frac{m n_0 v_{cm}^2}{3}
\]

\end{document}
