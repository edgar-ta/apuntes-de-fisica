\documentclass{article}
\usepackage{mathtools}
\usepackage{amsthm}
\usepackage{geometry}

\title{Ley General de los Gases Ideales}
\author{Edgar Trejo Avila}
\date{Miércoles 20 de octubre del 2022}

% https://www.overleaf.com/learn/latex/Theorems_and_proofs#Introduction
\newtheorem{law}{Ley}

% https://www.overleaf.com/learn/latex/Page_size_and_margins
\geometry{bmargin=1in,tmargin=1in}

\begin{document}
\maketitle

Considerando que \(n_0\) se puede
escribir como el número \(N\) de moléculas
por unidad \(V\) de volumen de gas (i. e., \(n_0 = \frac{N}{V}\)),
es posible escribir la ecuación de \emph{Gay Lussac}
como:

\begin{align*}
    P &= \frac{NkT}{V} \\
    \frac{PV}{T} &= Nk
\end{align*}

\begin{law}[Ley General de los Gases Ideales]
    En un gas ideal, \(\frac{PV}{T}\) es una constante,
    i. e., para una configuración inicial y final de un gas,
    se cumple que:

    \[
        \frac{P_iV_i}{T_i} = \frac{P_fV_f}{T_f}
    \]
\end{law}

\begin{law}[Ley de Boyle (transformación \textbf{isotérmica}\footnote{De la raíz griega \textit{iso-}, que significa \textit{igual}})]
    Cuando la temperatura de un gas es constante,
    se tiene que el producto de su presión y su volumen
    es constante, i. e.:

    \begin{align*}
        PV &= C\\
        P &= \frac{C}{V}
    \end{align*}
\end{law}

\begin{law}[Ley de Charles (transformación \textbf{isobárica}\footnote{De la raíz griega \textit{-baros}, que significa \textit{peso}})]
    Cuando la presión de un gas es constante,
    se tiene que el cociente de su volumen y
    su temperatura es constante, i. e.:

    \[
        \frac{V}{T} = \frac{Nk}{P} = C
    \]
\end{law}

\begin{law}[Ley de Gay Lussac (transformación \textbf{isocórica}\footnote{De la raíz griega \textit{-coros}, que significa \textit{espacio}})]
    Cuando el volumen de un gas es constante, 
    se tiene que el cociente de su presión y 
    su temperatura es constante, i. e.:

    \[
        \frac{P}{T} = \frac{Nk}{V} = C
    \]
\end{law}

% https://tex.stackexchange.com/a/192158
\thispagestyle{empty}

\end{document}
