\documentclass{article}
\usepackage{mathtools}
\usepackage{amsthm}
\usepackage{geometry}

\title{Leyes de los Gases 2}
\author{Edgar Trejo Avila}
\date{Jueves 10 de octubre del 2022}

% https://www.overleaf.com/learn/latex/Page_size_and_margins
\geometry{bmargin=1in,tmargin=1in}

\begin{document}
\maketitle

\section*{Ley de Charles (y Gay Lussac)}

En 1802, el físico francés L. Gay Lussac 
estudió la dilatación de los gases en función
de la temperatura cuando la presión se
mantiene constante (proceso \textbf{isobárico} \footnote{De las raíces griegas \textit{iso-}, que significa \textit{igual}, y \textit{-baros} que significa \textit{peso}}),
el inglés Luis A. C. Charles también obtuvo los
mismos resultados unos años antes.

\paragraph*{Something}

Como tal, la ley asegura que el cociente del 
volumen de un gas y su temperatura se mantienen
constantes, i. e.:



\end{document}
