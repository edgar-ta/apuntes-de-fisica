\documentclass{article}
\usepackage{mathtools}
\usepackage{amsthm}
\usepackage{geometry}

\title{La Dilatación Ideal}
\author{Edgar Trejo Avila}
\date{Sábado 29 de octubre del 2022}

% https://www.overleaf.com/learn/latex/Page_size_and_margins
\geometry{bmargin=1in,tmargin=1in}

\begin{document}
\maketitle

\section*{Dilatación Lineal}

Imagínese una barra de longitud inicial \(L_0\)
a una determinada temperatura inicial \(T_0\) que
hace que ésta (la barra) experimente un cambio de 
longitud (llamado "dilatación") representado por 
\(\Delta L\), de modo que su longitud final es \(L\).
Se sabe que \(\Delta L\) es directamente proporcional
al producto del cambio de temperatura \(\Delta T\) y 
\(L_0\), i.e., el cociente \(\frac{\Delta L}{L_0 \Delta T}\)
es una constante (llamada coeficiente de dilatación)
representada por la letra griega \(\alpha\), de modo que:

\begin{align}
    \Delta L = \alpha L_0 \Delta T \label{linearDilatation}
\end{align}

Esto es:

\begin{align*}
    L - L_0 &= \alpha L_0 \Delta T \\
    L       &= \alpha L_0 \Delta T + L_0 \\
            &= L_0 \left( \alpha \Delta T + 1 \right) 
\end{align*}

\section*{Dilatación Superficial}

Siguiendo con la premisa de la sección anterior, imagínese 
ahora un cuadrado de longitud \(L_0\) que experimenta una
dilatación. Es evidente que un cambio en su longitud desencadena
también uno en su área, y es de ahí que nace el propósito de 
esta sección, que es obtener una relación entre el área final 
y la inicial del cuadrado. 

Por la fórmula del área del cuadrado, se sabe que el área 
inicial \(A_0\) es igual a \(L_0^2\). De la misma manera,
el área final \(A\) es tal que:

\begin{align*}
    A   &= L^2 \\
        &= \left( L_0 + \Delta L \right)^2 \\
        &= L_0^2 + 2 L_0 \Delta L + \Delta L^2 \\
\end{align*}

Introduciendo a \(\Delta A\) como el cambio entre el
área inicial y la final:

\begin{align*}
    \Delta A    &= A - A_0 \\
                &= \left( L_0^2 + 2 L_0 \Delta L + \Delta L^2 \right) - A_0 \\
                &= \left( L_0^2 + 2 L_0 \Delta L + \Delta L^2 \right) - L_0^2 \\
                &= 2 L_0 \Delta L + \Delta L^2
\end{align*}

El cambio de longitud
es lo suficientemente pequeño como para ser
depreciable cuando su exponente es mayor a uno
(i. e., un cuadrado, un cubo, etc.), de modo que:

\[
    \Delta A = 2 L_0 \Delta L
\]

Luego:

\begin{align*}
    A   &= A_0 + \Delta A \\
        &= A_0 + 2 L_0 \Delta L \\
\end{align*}

Por (\ref{linearDilatation}):

\begin{align*}
    A   &= A_0 + 2 L_0 \Delta L \\
        &= A_0 + 2 L_0 \left( L_0 \alpha \Delta T \right) \\
        &= A_0 + 2 L_0^2 \alpha \Delta T \\
        &= A_0 + 2 A_0 \alpha \Delta T \\
        &= A_0 \left( 2 \alpha \Delta T + 1 \right)
\end{align*}

\section*{Dilatación Volumétrica}

Siguiendo con la misma premisa, uno puede calcular 
la relación entre el volumen inicial \(V_0\) y el
volumen final \(V\) de 
un cubo de longitud \(L_0\):

\begin{align*}
    V   &= L^3 \\
        &= \left( L_0 + \Delta L \right)^3 \\
        &= L_0^3 + 3 L_0^2 \Delta L + 3 L_0 \Delta L^2 + \Delta L^3 \\
        &= L_0^3 + 3 L_0^2 \Delta L \\
        &= L_0^3 + 3 L_0^2 \left( \alpha L_0 \Delta T \right) \\
        &= L_0^3 + 3 L_0^3 \alpha \Delta T \\
        &= V_0 + 3 V_0 \alpha \Delta T \\
        &= V_0 \left( 3 \alpha \Delta T + 1 \right) 
\end{align*}

Recuérdese que las potencias de \(\Delta L\) con exponente
mayor a uno se pueden depreciar.

\section*{Patrón General}

Como es fácil de ver, la fórmula para la dilatación 
sigue la estructura:

\begin{align}
    L_0^n \left( n \alpha \Delta T + 1 \right); \hspace{1cm} 1 \leq n \leq 3
\end{align}

Donde \(n\) representa la dimensión de la dilatación que 
se quiere medir


\end{document}
